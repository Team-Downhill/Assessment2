\section{Assignment}
This project involves designing and developing a functional BLDC motor controller with a user interface. The system must integrate sensors, a microcontroller, power electronics, and control algorithms to achieve efficient motor operation.

\subsection{Assignment Statement}
The goal of this project is to develop a BLDC motor controller that can drive a 24V, 78W BLDC motor in a closed-loop configuration using hall sensor feedback. The system must allow user control via a throttle, display real-time data, and be implemented on a custom PCB.

\subsection{Requirement}
The following requirements must be met.

% \section{Assignment}
% This project involves designing and developing a functional BLDC motor controller with a user interface. The system must integrate sensors, a microcontroller, power electronics, and control algorithms to achieve efficient motor operation.

\subsubsection{Business Requirements}
\begin{enumerate}[label=B\arabic*.]
    \item \textbf{Cost Constraints}: The final design should be developed within a budget of 100 euros (excluding the PCB).
    \item \textbf{Market Competitiveness}: The motor controller should enhance e-bike efficiency to attract customers willing to pay a premium.
    \item \textbf{Standardization}: The design should use industry-standard components available from approved suppliers (e.g., Conrad, Farnell).
    \item \textbf{Sustainability}: The final design should aim for energy efficiency to align with the theme of smart energy.
\end{enumerate}

\subsubsection{User Requirements}
\begin{enumerate}[label=U\arabic*.]
    \item \textbf{Ease of Use}: The user interface should allow intuitive control of motor speed via throttle input.
    \item \textbf{Performance Display}: The system must provide real-time feedback on motor speed (RPM) and current consumption.
    \item \textbf{Reliability}: The motor controller must operate continuously and reliably under standard e-bike conditions.
    \item \textbf{Safety}: The design must prevent hazardous conditions such as shoot-through in the transistors.
    \item \textbf{Compact and Durable}: The system should be integrated onto a single PCB, suitable for e-bike applications.
\end{enumerate}

\subsubsection{System Requirements}

\paragraph{Hardware Requirements}
\begin{enumerate}[label=H\arabic*.]
    \item \textbf{Motor Compatibility}: The controller must drive a \textbf{24V, 78W sensored BLDC motor} in a closed-loop system.
    \item \textbf{Microcontroller Selection}: The design must use either:
    \begin{itemize}
        \item \textbf{RP2040} (Raspberry Pi Pico) (Recommended)
        \item \textbf{STM32F103C8T6} (STM Blue Pill) (Alternative)
    \end{itemize}
    \item \textbf{Sensor Processing}: The hall sensor signals must be processed by the microcontroller (no advanced motor driver ICs handling hall sensors internally).
    \item \textbf{Power Electronics}: The system must use discrete transistors for the three-phase motor control.
    \item \textbf{PCB Integration}: The final circuit must be assembled on a custom PCB designed and produced by the team.
    \item \textbf{Current Measurement}: The controller must measure and report motor current consumption to the user interface.
\end{enumerate}

\paragraph{Software Requirements}
\begin{enumerate}[label=S\arabic*.]
    \item \textbf{Motor Control Algorithms}: The software must generate correct PWM signals to control motor speed and direction.
    \item \textbf{Dead-Time Implementation}: Software must ensure proper dead-time in PWM signals to prevent shoot-through.
    \item \textbf{User Input Handling}: The microcontroller must read throttle input and adjust motor speed accordingly.
    \item \textbf{Real-Time Data Processing}: The system must process hall sensor feedback to maintain accurate motor speed control.
    \item \textbf{System Initialization}: Define reset states for all IOs upon startup.
    \item \textbf{Fault Handling}: The software should detect and handle motor faults, such as overcurrent or sensor failures.
    \item \textbf{Modular and Maintainable Code}: The software must be structured, well-documented, and logically organized.
\end{enumerate}

\subsection{Project Boundaries}
The project includes:
\begin{itemize}
    \item Development of motor control software and firmware.
    \item Design and implementation of a PCB for the motor controller.
    \item Testing and validation of the system to ensure proper operation.
    \item Preparation of technical documentation and reports.
\end{itemize}
The project does not include:
\begin{itemize}
    \item Mass production or commercial deployment of the controller.
    \item Advanced features such as regenerative braking or wireless communication.
\end{itemize}

\subsection{Deliverables}
At the end of the project, the following must be completed:
\begin{itemize}
    \item Fully functional BLDC motor controller with user interface.
    \item Custom PCB with assembled components.
    \item Well-documented microcontroller code.
    \item Final report detailing the design, testing, and outcomes.
    \item Presentation and live demonstration of the final system.
\end{itemize}


\subsection{The Problem}
Op den Bergh Fietsen (ODB) has been selling low quality and inefficient electric bicycles, which used brushed DC motors. 




% \section{Project activities}
% As part of the project there will be multiple assessments. The goals for these assessments are set in advance, in this section all the assessment goals will be listed. 

% \section{Introduction}
% This project has been set up as part of the Smart Energy Devices course to design, build, and test a sensored BLDC motor controller for an electric bicycle. The project is structured as if each student is working within the electrical engineering team of a hypothetical e-bike design company, "Op den Bergh Fietsen." The team consists of three students, each contributing to different aspects of the project, including hardware, software, and documentation.

% \section{Assignment}
% This project involves designing and developing a functional BLDC motor controller with a user interface. The system must integrate sensors, microcontrollers, power electronics, and control algorithms to achieve efficient motor operation.

% \subsection{Assignment Statement}
% The goal of this project is to develop a BLDC motor controller that can drive a 24V, 78W BLDC motor in a closed-loop configuration using hall sensor feedback. The system must allow user control via a throttle, display real-time data, and be implemented on a custom PCB.

% \subsection{Project Boundaries}
% The project includes:
% \begin{itemize}
%     \item Development of motor control software and firmware.
%     \item Design and implementation of a PCB for the motor controller.
%     \item Testing and validation of the system to ensure proper operation.
%     \item Preparation of technical documentation and reports.
% \end{itemize}
% The project does not include:
% \begin{itemize}
%     \item Mass production or commercial deployment of the controller.
%     \item Advanced features such as regenerative braking or wireless communication.
% \end{itemize}

% \subsection{Deliverables}
% At the end of the project, the following must be completed:
% \begin{itemize}
%     \item Fully functional BLDC motor controller with user interface.
%     \item Custom PCB with assembled components.
%     \item Well-documented microcontroller code.
%     \item Final report detailing the design, testing, and outcomes.
%     \item Presentation and live demonstration of the final system.
% \end{itemize}
\section{Organisation}
\subsection{Task Division}
Each group member will have a defined role:
\begin{itemize}
    \item \textbf{Tommy}: Final responsibility for hardware design and PCB layout.
    \item \textbf{Jules}: Final responsibility for software development and firmware implementation.
    \item \textbf{Luca}: Final responsibility for testing, documentation, and report writing.
\end{itemize}
All members will collaborate on integration, testing, debugging, documenting, and final presentation preparation.

\subsection{Communication}
The team will communicate via:
\begin{itemize}
    \item Weekly in-person meetings to discuss progress and challenges.
    \item Online communication via WhatsApp.
    \item A shared Overleaf document for recording findings and updates.
\end{itemize}
Meeting notes and key decisions will be documented in a project log.

\subsection{Planning}
The project timeline is structured around the assessment deadlines:
\begin{itemize}
    \item \textbf{Second Assessment} (March 27, 17:00): Submission of microcontroller software design requirements, documented code, and initial testing summary.
    \item \textbf{In-person Code Review} (April 1, 8:45-11:45): Discussion and review of software implementation.
    \item \textbf{Third Assessment} (May 1, 17:00): Submission of PCB design documents, including schematics and layout.
    \item \textbf{In-person PCB Review} (May 6, 14:45-17:45): Discussion and review of the PCB design and its functionality.
    \item \textbf{Final Assessment} (June 19, 17:00): Submission of the final report, including project outcomes and individual competencies.
    \item \textbf{Final Presentation and Demo} (June 24, 8:45): Live demonstration and explanation of the final product.
\end{itemize}




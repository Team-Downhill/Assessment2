\section{Introduction}
This project has been set up as part of the Smart Energy Devices course to design, build, and test a sensored BLDC motor controller for an electric bicycle. The project is structured as if each student is working within the electrical engineering team of a hypothetical e-bike design company, "Op den Bergh Fietsen." The team consists of three students, each contributing to different aspects of the project, including hardware, software, and documentation.


\section{Background}
The company Op den Berg Fietsen has been selling low-quality electric bikes using brushed DC motors. The company, wanting to modernise, has decided to change all its new bikes over to using a brushless DC motor. To achieve this goal, the project: "Motor Driver" has been set up. The goal of this project is to design a PCB and accompanying PLC code to drive the motors that will be part of these new bikes.\\
\\
To complete this project, Team Downhill has been hired. This team consists of Jules van Velzen, Luca van den Brink, and
Tommy Dobos. The team will need to investigate the internal workings of the brushless DC motor and determine the required output to power the motor.







\begin{comment}
Infra Vroom is de naam van een dynamisch team bestaande uit vier teamleden, Francisco Ramirez, Laurens van der Drift, Tommy Dobos en Justin van der Reijden.

Er is een opdracht gegeven door Dhiraj Djairam om een met sensoren bestuurd zelfrijdende auto te bouwen. Om een beter overzicht te behouden van dit proces zal in dit verslag continu een documentatie van het proces bijgehouden worden. Het idee achter dit project is om het proces van het bouwen van de auto te begrijpen en eventuele obstakels te analyseren en te begrijpen. De documentatie van dit proces moet zodanig geanalyseerd worden dat bij het tegenkomen van een probleem het makkelijk omzeilt kan worden. Het doel van dit project is met name het zo veel mogelijk ontwikkelen van onze vaardigheden. Om de functionaliteit en een doeltreffend groepswerk te garanderen, moet een leider worden toegewezen. Daarom is Laurens van de Drift als leider toegewezen om het succes van het proces waarborgen. De taak van de projectleider is het proces in goede banen leiden en de groep tot het einde van het project begeleiden.
\end{comment}